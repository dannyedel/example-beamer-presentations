\documentclass{beamer}

\usepackage[utf8]{inputenc}
\usepackage[english]{babel}

\title{Greek Alphabet}
\subtitle{Demonstration presentation for usage with dspdfviewer}
\author{Benjamin Desef}
\date{11.08.2015}

\makeatletter

% The presentation and the notes will get a line stretch of this value
% Default is 1.3.
\def\globalSpacing{1.3}

% You may use the \themeoptions macro to append options to the Hannover theme.
% Default is hideothersubsections.
%\def\themeOptions{}

% This will control whether notes are suppressed (1) or not (0)
% Default is 0.
%\def\hideNotes{1}

\usepackage{etex}
\usepackage{lmodern,amssymb,amsmath,fourier,microtype,setspace,calc,pgfmorepages,zref-user,zref-savepos,tikz,xifthen,xstring}

\ifdefined\globalSpacing\relax\else\def\globalSpacing{1.3}\fi
\ifdefined\hideNotes\relax\else\def\hideNotes{0}\fi

% Setting up the twoside presentation
\newcounter{dupnext}
\pgfpagesdeclarelayout{dn}{}{
   \pgfpagesphysicalpageoptions{%
      logical pages=2,%
      physical height=\pgfpageoptiontwoheight,%
      physical width=\pgfpageoptiontwowidth,%
      last logical shipout=1%
   }
   \pgfpageslogicalpageoptions{1}{%
      center=\pgfpageoptionfirstcenter,%
      skip code={{%
         \hbadness=10000%
         \vbadness=10000%
         \parbox[l][0pt][t]{0pt}{%
            \vskip-\pgfphysicalheight%
            \vskip1mm%
            \hspace*{.5\pgfphysicalwidth}%
            \rule{1pt}{\pgfphysicalheight}%
         }%
      }},%
   }%
   \pgfpageslogicalpageoptions{2}{%
      center=\pgfpageoptionsecondcenter,%
      skip code={{%
         \hbadness=10000%
         \vbadness=10000%
         \ifthenelse{\value{dupnext}>0}{%
            \ifvoid\csname pgfpages@box@1\endcsname\else%
            \setbox\csname pgfpages@box@2\endcsname=%
            \hbox to \pgfphysicalwidth{%
               \hskip-.6in%
               \vbadness=10000
               \vbox to \pgfphysicalheight{%
                  \vskip-1in%
                  \parbox[l][0pt][t]{0pt}{%
                     \hfuzz=\maxdimen%
                     \vskip-\pgfphysicalheight%
                     \vskip3mm%
                     \parbox{2.5cm}{% Add an information that the next slide is shown
                        \hskip-2mm%
                        \textcolor{red}{Next slide}%
                     }%
                  }%
                  \copy\beamer@frameboxcopy%
               }%
            }\fi%
            \ifthenelse{\value{dupnext}=2}{%
               \setcounter{dupnext}{0}%
            }{%
               \ifthenelse{\value{dupnext}>2}{%
                  \addtocounter{dupnext}{-1}%
               }{}%
            }%
         }{%
            \setcounter{dupnext}{-\value{dupnext}}%
         }%
      }},%
   }%
}
\newenvironment{peek}{%
   \setcounter{dupnext}{-1}%
}{%
   \setcounter{dupnext}{0}%
}
\newcommand{\peeknext}[1][1]{%
   \ifthenelse{#1=-1}{%
      \setcounter{dupnext}{-1}%
   }{%
      \setcounter{dupnext}{-#1-1}%
   }%
}
\newcommand{\stoppeek}{%
   \setcounter{dupnext}{0}%
}
\setbeamertemplate{note page}[plain]
\if\hideNotes1
   \setbeameroption{hide notes}
\else
   \pgfpagesuselayout{dn}[second right]
   \beamer@twoscreensnotestrue
   \beamer@notestrue
   \newcounter{pageanchored}
   \def\Hy@EveryPageAnchor{%
      \ifthenelse{\value{pageanchored}<\the\c@page}{% Don't insert twice
         \Hy@DistillerDestFix
         \vbox to 0pt{%
            \kern\voffset
            \kern\topmargin
            \kern-1bp\relax
            \hbox to 0pt{%
               \kern\hoffset
               \kern\oddsidemargin
               \kern-1bp\relax
               \hyper@@anchor{Navigation\the\c@page}\relax%
               \hyper@@anchor{page.\the\c@page}\relax%
               \hss
            }%
            \vss
         }%
         \setcounter{pageanchored}{\the\c@page}%
      }{}
   }
\fi

% Add spacing to every slide
\let\oldbegin\beamer@framenotesbegin
\def\beamer@framenotesbegin{\setstretch{\globalSpacing}\oldbegin}

% Add indention capability to navigation for subsections
\newlength{\sAw}
\newlength{\sfl}
\newcommand{\isec}[3]{% #1: Split char, #2: String to insert, #3: Beamer class
   \StrCut{#2}{#1}{\sA}{\sB}%
   \ifthenelse{\equal{\sB}{}}{% \isempty does not work
      \beamer@sidebarformat{5pt}{#3}{%
         \sA%
      }%
   }{%
      \setlength{\sfl}{5pt}%
      \settowidth{\sAw}{\sA#1}%
      \addtolength{\sfl}{\sAw}%
      \beamer@sidebarformat{\sfl}{#3}{%
         \hspace{-\sAw}%
         \sA#1\sB%
      }%
   }%
}

% Replace note command by the special, aligning one with hyperlink support
\let\oldnote\note
\if\hideNotes1
   \AtBeginNote{%
      \setstretch{\globalSpacing}%
   }
   \renewcommand<>{\note}[1]{%
      \only#2{%
         \vbox to 0pt{}% Same spacing as with notes
      }%
   }
\else
   \newcounter{pposes}
   \newcounter{poses}
   \AtBeginNote{%
      \setcounter{poses}{0}%
      \setstretch{\globalSpacing}%
      \let\hyper@link@\hyper@link@note
   }
   \AtEndNote{%
      \setcounter{poses}{0}%
      \stepcounter{pposes}%
      \let\hyper@link@\hyper@link@default
   }
   \renewcommand<>{\note}[1]{%
      \only#2{%
         \stepcounter{poses}%
         \vbox to 0pt{%
            \edef\l{p\the\value{pposes}-\the\value{poses}}%
            \zsaveposy{\l}%
            \zrefused{\l}%
            \oldnote{%
               \stepcounter{poses}%
               \edef\l{p\the\value{pposes}-\the\value{poses}}%
               \zref@ifrefundefined{\l}{}{%
                  \parbox[l][0pt][t]{0pt}{%
                     \hfuzz=\maxdimen%
                     \parbox{\textwidth}{%
                        \vspace{\dimexpr\paperheight - \zposy{\l}sp - 6mm}%
                        #1%
                     }%
                  }%
               }%
            }%
         }%
      }%
   }
   \let\hyper@link@default\hyper@link@
   \newcommand\hyper@link@note[4][]{%
      \begingroup%
         \parbox[l][0pt][t]{0pt}{%
            \hfuzz=\maxdimen%
            \vspace*{-1.5mm}%
            \hspace*{\paperwidth}%
            \hyper@link@default[#1]{#2}{#3}{\phantom{#4}}%
         }%
         \color{blue}#4%
      \endgroup%
   }
\fi
\newcommand<>{\Note}[1]{%
   \vspace{-\baselineskip}%
   \note#2{#1}%
}
\newcommand<>{\secEnd}[1][]{%
   \only#2{\oldnote{%
      \parbox[l][0pt][t]{0pt}{%
         \hfuzz=\maxdimen%
         \parbox{\textwidth}{%
            \hfuzz=\maxdimen%
            \vspace{\dimexpr\paperheight - 2.35cm}%
            \hspace{\dimexpr\paperwidth - 3cm}%
            \begin{tikzpicture}[baseline=(current bounding box.south)]
               \draw (0, 0) -- (2, 2)
                     (.05, 0) -- (2, 1.95)
                     (.1, 0) -- (2, 1.9);
               \begin{pgfinterruptboundingbox}
                  \node[rotate=45, anchor=north, align=center] at (1, .95) {#1};
               \end{pgfinterruptboundingbox}
            \end{tikzpicture}
         }%
      }%
   }}%
}

% Set some cosmetics defaults
\ifdefined\themeOptions\relax\else\def\themeOptions{hideothersubsections}\fi
% Set PDF information
\hypersetup{
   pdfstartview={Fit},
   pdftitle={\@title},
   pdfauthor={\@author},
   hidelinks,
   breaklinks=false,
}

% Fix bad bracket size in Fourier
\renewcommand{\big}{\bBigg@{1}}
\renewcommand{\Big}{\bBigg@{1.5}}
\renewcommand{\bigg}{\bBigg@{2.4}}
\renewcommand{\Bigg}{\bBigg@{3.2}}
% Replace bad Fourier arrows by those in Latin Modern
\DeclareSymbolFont{lmsymbols}{OMS}{lmsy}{m}{n}
\DeclareMathSymbol{\Leftarrow}{\mathrel}{lmsymbols}{40}
\DeclareMathSymbol{\Rightarrow}{\mathrel}{lmsymbols}{41}
\DeclareMathSymbol{\Leftrightarrow}{\mathrel}{lmsymbols}{44}

% Other presentation settings
\usetheme[width=1.95cm, \themeOptions]{Hannover}
\usefonttheme[onlymath]{serif}
\setbeamercolor*{section in sidebar shaded}{parent=palette sidebar secondary}
\setbeamercolor*{section in sidebar}
  {parent=section in sidebar shaded,use={sidebar,section in sidebar shaded},%
   bg=sidebar.bg!70!section in sidebar shaded.fg}
\setbeamercolor*{subsection in sidebar shaded}{parent=palette sidebar primary}
\setbeamercolor*{subsection in sidebar}
  {parent=subsection in sidebar shaded,use=section in sidebar,%
   bg=section in sidebar.bg}
\beamertemplatedotitem
\setbeamertemplate{frametitle}[default][left]
\setbeamertemplate{navigation symbols}{}
\setbeamertemplate{itemize subitem}[triangle]
\setbeamertemplate{itemize subsubitem}[ball]

% Indent sections
\newcommand{\indentSection}[1]{
   \setbeamertemplate{section in sidebar}{%
      \isec{#1}{\insertsectionhead}{section in sidebar}%
   }
   \setbeamertemplate{section in sidebar shaded}{%
      \isec{#1}{\insertsectionhead}{section in sidebar shaded}%
   }
}
\newcommand{\indentSubsection}[1]{
   \setbeamertemplate{subsection in sidebar}{%
      \isec{#1}{\insertsubsectionhead}{subsection in sidebar}%
   }
   \setbeamertemplate{subsection in sidebar shaded}{%
      \isec{#1}{\insertsubsectionhead}{subsection in sidebar shaded}%
   }
}
\newcommand{\indentSubsubsection}[1]{
   \setbeamertemplate{subsubsection in sidebar}{%
      \isec{#1}{\insertsubsubsectionhead}{subsubsection in sidebar}%
   }
   \setbeamertemplate{subsubsection in sidebar shaded}{%
      \isec{#1}{\insertsubsubsectionhead}{subsubsection in sidebar shaded}%
   }
}

% Math mode line spacing
\setlength{\jot}{2mm}

% Use those commands to define whether the navigation titels contain enumeration items.
% For those, indention is controlled so they are aligned properly. The commands expect
% the delimiter as a first parameter.
%\indentSection{ }
\indentSubsection{ }
%\indentSubsubsection{ }

\makeatother

\begin{document}
   \maketitle
   \section{Majuscules}
      \subsection{§1 First eight capital Greek letters}
         \begin{frame}{Capital letters}{First eight}
            \begin{enumerate}
               \item Alpha, Beta
               \item Gamma, Delta
                  \note{This is an annotation that belongs to ``Gamma, Delta''.}
               \item Epsilon, Zeta
               \item Eta, Theta
            \end{enumerate}
         \end{frame}
         \subsubsection{Alpha, Beta}
            \begin{frame}{Capital letters}{Alpha, Beta}
               \[ A \note{Annotations will always be inserted at the same $y$ position as the text they belong to. Insertion into math mode is also possible (align environments as well as \texttt{\textbackslash[ \textbackslash]}).} \]
               \[ B \]
            \end{frame}
         \subsubsection{Gamma, Delta}
            \begin{frame}{Capital letters}{Gamma, Delta}%
               % The capital \Note command is needed, as \note will start a new paragraph. If no text follows, this leads to a wrong y position of the equations. I have not found a way to avoid this; basically, as soon as \pdfsavepos is triggered (here by \zsaveposy), the paragraph will begin. \Note compensates this by inserting a negative \vspace (undesirable if text follows).
               \Note{Note that in the navigation bar, the long title of this subsection is adjusted in a way that the enumeration is aligned properly. For this purpose, the caption is split at a defined delimiter (which can easily be adjusted via the parameter of the \texttt{\textbackslash indent[Sub[Sub]]Section} command, see lines 31\,--\,33). This behavior is currently implemented for subsections, but can easily be extended to sections and subsubsections.}
               \[ \Gamma \]
               \[ \Delta \]
            \end{frame}
         \subsubsection{Epsilon, Zeta}
            \begin{frame}[label=ce]{Capital letters}{Epsilon, Zeta}
               \[ E \note{Jump to \hyperlink{se}{small epsilon} -- hyperlinks work in annotations as well.} \]
               \[ Z \]
            \end{frame}
         \subsubsection{Eta, Theta}
            \begin{frame}{Capital letters}{Eta, Theta}
               \[ H \note{The next slide will show a preview of the following slide in the internal area. This can be done by enclosing the frames in a \texttt{peek} environment (but it cannot be triggered on or of \emph{within} one frame). It works for overlays as well as for more than one frame. You should not make use of notes on slide which has this key.} \]
               \[ \Theta \]
            \end{frame}
      \subsection{§2 Next eight capital Greek letters}
         \begin{peek}
            \begin{frame}{Capital letters}{Next eight}
               \begin{enumerate}[<+>]
                  \setbeamercovered{transparent=50}
                  \item Iota, Kappa
                  \item Lambda, Mu
                  \item Nu, Xi
                  \item Omicron, Pi
               \end{enumerate}
            \end{frame}
         \end{peek}
         \subsubsection{Iota, Kappa}
            \begin{frame}{Capital letters}{Iota, Kappa}
               \[ I \note{If you do only want to show the next slide on the right side, you can use \texttt{\textbackslash peeknext} before the frame that should contain the duplication. The macro accepts an optional parameter which should be a positive integer (default $1$) that indicates for how long this state should be kept. Set it to $-1$ for infinite peeking (which can be stopped with \texttt{\textbackslash stoppeek}). This is demonstrated in the next frame.} \]
               \[ K \]
            \end{frame}
         \peeknext
         \subsubsection{Lambda, Mu}
            \begin{frame}{Capital letters}{Lambda, Mu}
               \[ \Lambda \]
               \[ M \]
            \end{frame}
         \subsubsection{Nu, Xi}
            \begin{frame}{Capital letters}{Nu, Xi}
               \[ N \note{Sometimes, you may find it useful to indicate the end of a specific section in your notes. You can use the command \texttt{\textbackslash secEnd} for this, which accepts an optional parameter for a text that can be added to the end mark. This is demonstrated in the next slide (look at the bottom right).} \]
               \[ \Xi \]
            \end{frame}
         \subsubsection{Omicron, Pi}
            \begin{frame}{Capital letters}{Omicron, Pi}
               \[ O \]
               \[ \Pi \]
            \end{frame}
            \secEnd[End of §2 \\ \scriptsize Next: §3]
      \subsection{§3 Last eight capital Greek letters}
         \begin{frame}{Capital letters}{Last eight}
            \begin{enumerate}[<+->]
               \item Rho, Sigma \note<1>{Notes can also be bound to overlays.}
               \item Tau, Upsilon \note<2>{Those overlays can be specified directly at the note command: \texttt{\textbackslash note<1->\{...\}}.}
               \item Phi, Chi \note<3>{But it is also possible to include nodes in outer overlay commands.}
               \item Psi, Omega \note<4>{Just keep in mind that any of the commands that only remove the ink, but not the spacing (such as \texttt{\textbackslash uncover}) will have no effect on notes. Only commands such as \texttt{\textbackslash only} will work as expected.}
            \end{enumerate}
         \end{frame}
         \subsubsection{Rho, Sigma}
            \begin{frame}{Capital letters}{Rho, Sigma}
               \[ P \]
               \[ \Sigma \]
               \secEnd[\tiny end of comments]
            \end{frame}
         \subsubsection{Tau, Upsilon}
            \begin{frame}{Capital letters}{Tau, Upsilon}
               \[ T \]
               \[ \Upsilon \]
            \end{frame}
         \subsubsection{Phi, Chi}
            \begin{frame}{Capital letters}{Phi, Chi}
               \[ \Phi \]
               \[ X \]
            \end{frame}
         \subsubsection{Psi, Omega}
            \begin{frame}{Capital letters}{Psi, Omega}
               \[ \Psi \]
               \[ \Omega \]
            \end{frame}
   \section{Minuscules}
      \subsection{§1 First eight Greek letters}
         \begin{frame}{Small letters}{First eight}
            \begin{enumerate}
               \item Alpha, Beta
               \item Gamma, Delta
               \item Epsilon, Zeta
               \item Eta, Theta
            \end{enumerate}
         \end{frame}
         \subsubsection{Alpha, Beta}
            \begin{frame}{Small letters}{Alpha, Beta}
               \[ \alpha \]
               \[ \beta \]
            \end{frame}
         \subsubsection{Gamma, Delta}
            \begin{frame}{Small letters}{Gamma, Delta}
               \[ \gamma \]
               \[ \delta \]
            \end{frame}
         \subsubsection{Epsilon, Zeta}
            \begin{frame}[label=se]{Small letters}{Epsilon, Zeta}
               \[ \epsilon \quad \varepsilon \note{Back to \hyperlink{ce}{capital epsilon}.} \]
               \[ \zeta \]
            \end{frame}
         \subsubsection{Eta, Theta}
            \begin{frame}{Small letters}{Eta, Theta}
               \[ \eta \]
               \[ \theta \quad \vartheta \]
            \end{frame}
      \subsection{§2 Next eight Greek letters}
         \begin{frame}{Small letters}{Next eight}
            \begin{enumerate}
               \item Iota, Kappa
               \item Lambda, Mu
               \item Nu, Xi
               \item Omicron, Pi
            \end{enumerate}
         \end{frame}
         \subsubsection{Iota, Kappa}
            \begin{frame}{Small letters}{Iota, Kappa}
               \[ \iota \]
               \[ \kappa \]
            \end{frame}
         \subsubsection{Lambda, Mu}
            \begin{frame}{Small letters}{Lambda, Mu}
               \[ \lambda \]
               \[ \mu \]
            \end{frame}
         \subsubsection{Nu, Xi}
            \begin{frame}{Small letters}{Nu, Xi}
               \[ \nu \]
               \[ \xi \]
            \end{frame}
         \subsubsection{Omicron, Pi}
            \begin{frame}{Small letters}{Omicron, Pi}
               \[ o \]
               \[ \pi \quad \varpi \]
            \end{frame}
      \subsection{§3 Last eight capital Greek letters}
         \begin{frame}{Small letters}{Last eight}
            \begin{enumerate}
               \item Rho, Sigma
               \item Tau, Upsilon
               \item Phi, Chi
               \item Psi, Omega
            \end{enumerate}
         \end{frame}
         \subsubsection{Rho, Sigma}
            \begin{frame}{Small letters}{Rho, Sigma}
               \[ \rho \quad \varrho \]
               \[ \sigma \quad \varsigma \]
            \end{frame}
         \subsubsection{Tau, Upsilon}
            \begin{frame}{Small letters}{Tau, Upsilon}
               \[ \tau \]
               \[ \upsilon \]
            \end{frame}
         \subsubsection{Phi, Chi}
            \begin{frame}{Small letters}{Phi, Chi}
               \[ \phi \quad \varphi \]
               \[ \chi \]
            \end{frame}
         \subsubsection{Psi, Omega}
            \begin{frame}{Small letters}{Psi, Omega}
               \[ \psi \]
               \[ \omega \]
            \end{frame}
\end{document}